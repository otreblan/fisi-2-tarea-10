\makeatletter
\def\input@path{{../}}
\makeatother
\documentclass[../main.tex]{subfiles}

\graphicspath{{ima/problemas}{../ima/problemas}}

% Aquí empieza el documento{{{
\begin{document}
\section{Problemas}%

\thispagestyle{fancy}

\subsection{}%

En el siguiente circuito $RCL$ en serie,
el voltímetro $V_1$ registra una medida de $85V$.
Todos los instrumentos mostrados son ideales.
Si el periodo de la fuente alterna es de $20ms$, halla:

\begin{figure}[H]
	\centering
	\begin{tikzpicture}
		[
			circuit ee IEC,
			set resistor graphic   = var resistor IEC graphic,
			circuit declare symbol = multimeter,
			set multimeter graphic =
			{
				draw,
				inner sep = 3pt,
				shape     = generic circle IEC
			}
		]
		\draw (0,0)
			to
			[
				capacitor  = {pos=0.5, info'=$30\mu F$},
				multimeter = {pos=0.75, node contents=$A$}
			]
			++(4,0)
			to[resistor={info'=$150\Omega$}]
			++(0,3)
			to
			[
				contact  = {pos=0.1, name={contact 1}},
				inductor = {info'=$250mH$},
				contact  = {pos=0.9, name={contact 2}}
			]
			++(-4,0)
			to
			[
				contact   = {pos=0.1, name={contact 3}},
				ac source = {info=0:$V(t)$, rotate=90},
				contact   = {pos=0.9, name={contact 4}}
			]
			(0,0)
			;

		\node (v1) [multimeter]
			at ($(contact 3)!0.5!(contact 4)+(180:1.25)$) {$V_1$};
		\draw (v1.north) |- (contact 3);
		\draw (v1.south) |- (contact 4);

		\node (v2) [multimeter]
			at ($(contact 1)!0.5!(contact 2)+(90:1.25)$) {$V_2$};
		\draw (v2.east) -| (contact 1);
		\draw (v2.west) -| (contact 2);

	\end{tikzpicture}
\end{figure}

\begin{enumerate}[label=\alph*)]
	\item La impedancia total del circuito.
		\begin{align*}
			X_L &= 50*2\pi*250*10^{-3}\\
			X_L &= 125i\\
			\\
			X_C &= \frac{1}{50*2\pi*30*10^{-6}}\\
			X_C &= -\frac{1000}{3\pi}i\\
			\\
			Z &= X_L+X_C+R\\
			Z &= (150 +\frac{375\pi-1000}{3\pi} i)\Omega\\
			Z &= 151.19\Omega\phase{7.18°}
		\end{align*}
	\item La lectura del voltímetro $V_2$.
		\begin{align*}
			I &= \frac{V}{Z}\\
			I &= \frac{85V\phase{0°}}{151.19\Omega\phase{7.18°}}\\
			I &= 0.56A\phase{-7.18°}\\
			\\
			V_1 &= I*X_L\\
			V_1 &= 0.56\phase{-7.18°}*125\phase{90°}\\
			V_1 &= 70V\phase{82.82°}
		\end{align*}
	\item La lectura del amperímetro $A_1$.
		\begin{align*}
			I &= 0.56A\phase{-7.18°}
		\end{align*}
	\item El ángulo de desfase entre el voltaje y la corriente.
		¿La corriente está atrasada o adelantada respecto al voltaje?
		\begin{align*}
			V_\theta &= 0°\\
			I_\theta &= -7.18°\\
		\end{align*}
		Atrasada
\end{enumerate}

\subsection{}%

Se utiliza un osciloscopio para medir el voltaje en una fuente alterna,
obteniéndose el siguiente resultado.
Determina los valores de:

\begin{figure}[H]
	\centering
	\begin{tikzpicture}[domain=-2.5:12.5, scale=0.6]

		\boldmath
		\draw[<->] (-4,0) -- (14,0) node[right] {$t(ms)$};
		\draw[<->] (0,-6) -- (0,6) node[above] {Voltaje $(V)$};

		\draw
		[
			ultra thick,
			color   = blue,
			samples = 200
		]
		plot (\x,{5.20*sin(((\x+2.5)*(pi/(7.5)))r)});

		\node [below] at(-2.5,0) {$-2.5$};
		\node [left] at(0,4.5) {$4.5$};
		\node [below] at(5,0) {$5.0$};
	\end{tikzpicture}
\end{figure}


\begin{enumerate}[label=\alph*)]
	\item El periodo de la onda.
		\subitem $15ms$
	\item El voltaje pico.
		\begin{align*}
			V(0) &= 4.5\\
			k*\sen
			{
				\left(
					(0+2.5)*
					\left(
						\frac{\pi}{7.5}
					\right)
				\right)
		} &= 4.5\\
		k*\sen
		{
			\left(
				\frac{\pi}{3}
			\right)
		} &= 4.5\\
		k* \frac{\sqrt{3}}{2}  &= 4.5\\
		k &= 3\sqrt{3}V = V_p
		\end{align*}
	\item El voltaje $RMS$.
		\begin{align*}
			V_p * \sqrt{2}\\
			3\sqrt{5}V
		\end{align*}
	\item El voltaje medio.
		\subitem $0V$
\end{enumerate}


\end{document}
%}}}
